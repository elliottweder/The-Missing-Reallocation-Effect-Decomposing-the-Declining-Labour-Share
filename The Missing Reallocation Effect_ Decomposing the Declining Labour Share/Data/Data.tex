The aggregate labour share in year $t$, $\lambda_{t}$, can be written as the weighted sum of $N$ sectoral labour shares
\begin{equation}
    \lambda_{t} = \sum_{i=1}^{N}\omega_{it}\lambda_{it}.
\label{eqn:weighted_ls}
\end{equation}
The sectoral weights and labour shares are defined as
\begin{equation}
    \omega_{it} = \frac{VA_{it}}{\sum_{i=1}^{N}VA_{it}}
\end{equation}
and 
\begin{equation}
   \lambda_{it} = \frac{WL_{it}}{VA_{it}} 
\end{equation}
in which $VA_{it}$ and $WL_{it}$ denote sector $i$'s gross value-added and labour income, respectively, in year $t$. I use value-added data from the Bureau of Economic Analysis (BEA) industry accounts. Labour income in sector $i$ in year $t$ equals the sum of payroll workers' labour income $WL_{it}^{p}$ and self-employed persons' labour incomes $WL_{it}^{s}$. The first is measured using payroll tax records. However, the labour-capital split is not recorded for self-employed persons' total income. How should labour income be divided from returns on capital investments for self-employed individuals? Following the literature, I exploit four different assumptions for imputing self-employed labour income, resulting in the four labour share definitions I use throughout the paper. 

The first is the `same-wage-distribution' approach used in the BEA-BLS integrated industry-level production account \citep{eldridgeBEABLSIntegratedIndustrylevel2020}. First, assume the average wages for self-employed individuals and payroll workers are equal within skill$\times$demographic cells $j$, for each sector $i$ and year $t$. Second, the labour income $WL_{jit}^{s}$ of self-employed individuals equals the labour income of payroll workers $WL_{jit}^{p}$ corrected by the ratio of hours worked (which are recorded for both)
\begin{equation}
    WL_{jit}^{s} = WL_{jit}^{p} \times \frac{L_{jit}^{s}}{L_{jit}^{p}}
\end{equation}
in which $L_{jit}^{p}$ and $L_{jit}^{p}$ correspond to the total hours worked for payroll workers and self-employed individuals in cell $j$, sector $i$ and year $t$, respectively. Second, the `same-labour-share' approach by \citet{mendieta-munozDeclineUSLabor2021} corrects total self-employed individuals' income $Y_{it}^{s} \equiv WL_{it}^{s} + RK_{it}^{s}$ (the labour-capital split is not known) by the payroll labour share $\frac{WL_{it}^{p}}{VA_{it}}$
\begin{equation}
    WL_{it}^{s} = \frac{WL_{it}^{p}}{VA_{it}} \times Y_{it}^{s}.
\label{eqn:same_labour_share}
\end{equation}
Third, the `economy-wide labour share' assumption subtracts the total self-employed persons' income from the denominator in the payroll labour share in equation (\ref{eqn:same_labour_share})
\begin{equation}
    WL_{it}^{s} = \frac{WL_{it}^{p}}{VA_{it} - Y_{it}^{s}} \times Y_{it}^{s}.
\label{eqn:economy_wide}
\end{equation}
Lastly, since the first three definitions impute self-employed persons' labour income, they are subject to potential measurement error. To avoid mismeasurement, I examine the payroll labour share $\frac{WL_{it}^{p}}{VA_{it}}$. The payroll labour share represents the part of the aggregate labour share which is unambiguously labour income. 

The final choice required to specify $\omega_{it}$ and $\lambda_{it}$ is the level at which sectors $i$ are defined. Aggregating the economy into fewer sectors reduces the extent to which reallocation can affect the aggregate labour share. For example, reallocation across NAICS 3-digit industries is always measured as a within-sector contribution if the industries are within the same NAICS 2-digit classification, even if the labour shares are constant within the 3-digit industries\footnote{Appendix \ref{fig:minormajor} demonstrates that aggregating NAICS 3-digit sectors into NAICS 2-digit classifications eliminates the entire between-sector effect, even though the effect is positive at the 2-digit level.}. Therefore, aggregating the economy into coarser sectors loads onto the within-sector contribution. Ideally, I would use establishment-level information, but this kind of data is generally not available for the universe of firms in the US economy, and only within certain major sectors. To compromise, and for comparison to previous studies, I use sectors classified at the NAICS 3-digit classification for the `same-wage-distribution' and `payroll labour share' measures. For the `same-labour-share' and `economy-wide' assumptions, I can only use data aggregated into 14 sectors because total self-employed income $Y_{it}^{s}$ is not measured at a more disaggregated level. Table \ref{tab:data} shows how the average sectoral labour share falls from the first 5 years to the last 5 years for each of the four labour share definitions. 


\begingroup
\renewcommand*{\arraystretch}{1.3}

\begin{table}[h]
    \centering
    \small
    \caption{\normalsize Average sectoral labour share for each definition.}
    \begin{tabular}{l|cccc}
        
        \toprule[1.1pt]
        
        Measure & Mean (first five) & Mean (last five) & Years & Number of sectors \\

        \midrule[1.1pt]

        Same-wage-distribution & 0.69 & 0.61 & 1987-2019 & 60 \\

        Same-labour-share & 0.69 & 0.65 & 1987-2017 & 14 \\

        Economy-wide & 0.78 & 0.68 & 1987-2017 & 14 \\

        Payroll & 0.65 & 0.62 & 1987-2011 & 60 \\

        \bottomrule[1.1pt]
    \end{tabular}

\begin{minipage}{\linewidth}
\captionsetup{justification=raggedright,singlelinecheck=false}
    \caption*{Source: see Appendix \ref{sec: data source} for the data sources of each measure.}
\end{minipage}
    \label{tab:data}
\end{table}

\endgroup
