The absence of a consensus view on the causes behind the falling labour share means economic models have to parse out different mechanisms that may be responsible. My results imply the models should incorporate all three quantitatively relevant channels - reallocation, within-sector mechanisms and co-movement of the two. How could models do this? What sort of mechanisms can jointly lead to reallocation between sectors and changing labour shares within sectors? Here I will discuss how the mechanisms may work in a model. 

First, reallocation between sectors may arise due to shifting consumption, investment, or intermediate input production patterns. Multi-sector models, with exogenous or endogenous production networks, such as those in \citet{gagglStructuralChangeProduction2023} and \citet{grassiIOIOSize2017}, capture all three sources but have not been applied to studies of the labour share. 

Second, co-movements in sectoral weights and labour shares may be incorporated by adding rising profit shares, outsourcing to domestically-produced intermediate inputs, or the adoption of automation into the model. Rising profit shares within industries reduce labour shares \citep{autorFallLaborShare2020, barkaiDecliningLaborCapital2020} and, at the same time, reallocate demand away from sectors where markups, and, hence, prices rose the most. Next, increased outsourcing (in terms of value) to domestically produced intermediate inputs generates growth in sectors central to the production network \citep{gagglStructuralChangeProduction2023}. At the same time, the increased outsourcing in terms of value contributes to a fall in the labour share within sectors since the cost share of labour falls \citep{castro-vincenziIntermediateInputPrices2022}. Lastly, the adoption of automation leads to a lower labour share within firms and sectors but may cause faster growth in industries where automation benefits are higher \citep{acemogluAutomationNewTasks2019}. To examine the mechanisms potentially at play, a full-scale model would be necessary, which is not the goal of this paper. 

