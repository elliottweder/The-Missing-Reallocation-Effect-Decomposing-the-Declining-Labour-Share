How much of the declining labour share arises due to reallocation between sectors with different labour share levels and how much comes from labour shares changing within sectors? I use both the shift-share and the Haltiwanger decomposition to decompose changes in the aggregate labour share. Most studies of the declining labour share use the shift-share method, whereas the Haltiwanger decomposition has not been used to the best of my knowledge.  

Using the weighted-average definition of the labour share in equation (\ref{eqn:weighted_ls_second}), the shift-share method decomposes changes in the labour share, $\Delta \lambda_{t} = \lambda_{t} - \lambda_{t-1}$, into a term capturing reallocation, arising from $\Delta\omega_{it}$, and a term capturing the effect of changing sectoral labour shares, derived from $\Delta\lambda_{it}$.
\begin{definition}[Shift-Share Decomposition]
    The change in the labour share, $\Delta \lambda_{t} = \lambda_{t} - \lambda_{t-1}$, can be decomposed into two terms
    \begin{equation}
        \Delta \lambda_{t} = \underbrace{\sum_{i=1}^{N}\Delta\omega_{it}\tilde{\lambda}_{it}}_\text{Between} + \underbrace{\sum_{i=1}^{N}\tilde{\omega}_{it}\Delta\lambda_{it}}_\text{Within} 
    \label{eqn:shift_share_decomposition}
    \end{equation}    
    in which $\Delta x_{it} = x_{it} - x_{it-1}$ and $\tilde{x}_{it} = \ddfrac{x_{it-1} + x_{it}}{2}$.
\end{definition}
\noindent Both terms in the shift-share decomposition keep the bases defined at their arithmetic means $\tilde{\lambda}_{it}$ and $\tilde{\omega}_{it}$, respectively. Alternatively, the Haltiwanger decomposition uses bases defined at their $t-1$ values and, therefore, splits changes in the aggregate labour share into a between, within, and cross term. 

\begin{definition}[Haltiwanger Decomposition]
    The change in the labour share, $\Delta \lambda_{t} = \lambda_{t} - \lambda_{t-1}$, can be decomposed into three terms
    \begin{equation}
        \Delta \lambda_{t} = \underbrace{\sum_{i=1}^{N}\Delta\omega_{it}\lambda_{it-1}}_\text{Between} + \underbrace{\sum_{i=1}^{N}\omega_{it-1}\Delta\lambda_{it}}_\text{Within} + \underbrace{\sum_{i=1}^{N}\Delta\omega_{it}\Delta\lambda_{it}}_\text{Cross}
    \label{eqn:haltiwanger_decomposition}
    \end{equation}
    in which $\Delta x_{it} = x_{it} - x_{it-1}$.
\end{definition}
\noindent Using either decomposition, the `Between' term is used to measure the effect of reallocation between sectors with different labour shares. The `Within' terms capture the effect of changing labour shares within industries. The `Cross' term in the Haltiwanger decomposition captures the impact of co-movement of sectoral weights and labour shares. Theorem \ref{theorem_1} demonstrates how the two decompositions are related since they are both exact decompositions of changes in the labour share $\Delta\lambda_{t}$. 

\begin{theorem}
\label{theorem_1}
By adding half of the Haltiwanger cross term to the Haltiwanger between-sector term, you get the shift-share between term.
\begin{equation*}
    \underbrace{\sum_{i=1}^{N}\Delta \omega_{it} \tilde{\lambda}_{it}}_\text{Shift-Share Between} = \underbrace{\sum_{i=1}^{N}\Delta  \omega_{it} \lambda_{it-1}}_\text{Haltiwanger Between} + \frac{1}{2}\underbrace{\sum_{i=1}^{N}\Delta \omega_{it} \Delta \lambda_{it}}_\text{Haltiwanger Cross}.
\end{equation*}
The same relationship holds for the within-sector terms.
\begin{equation*}
    \underbrace{\sum_{i=1}^{N}\tilde{\omega}_{it}\Delta \lambda_{it}}_\text{Shift-Share Within} = \underbrace{\sum_{i=1}^{N}\omega_{it-1}\Delta \lambda_{it}}_\text{Haltiwanger Within} + \frac{1}{2}\underbrace{\sum_{i=1}^{N}\Delta \omega_{it} \Delta \lambda_{it}}_\text{Haltiwanger Cross}.     
\end{equation*}
\end{theorem}
\begin{proof}
    See Appendix \ref{sec: theorem_1_proof}.
\end{proof}
\noindent Theorem \ref{theorem_1} demonstrates the shift-share between-sector term, which aims to capture the effect of reallocation, also captures half of the impact of co-movements between $\omega_{it}$ and $\lambda_{it}$. The same logic applies to the within-sector terms. Moreover, a non-zero Haltiwanger cross term leads to different conclusions about the impact of between-sector reallocation and within-sector mechanisms depending on the decomposition method used. For example, suppose the Haltiwanger between-sector term is positive, which means reallocation leads to a counterfactually higher aggregate labour share. Using the shift-share decomposition instead can result in a zero reallocation effect being measured if the Haltiwanger cross term is large and negative, because it subtracts from the positive Haltiwanger between-sector term, via Theorem \ref{theorem_1}. Additionally, the negative cross term leads to negative within-sector mechanisms being overcounted. Therefore, the Haltiwanger decomposition provides a more suitable measure of reallocation and within-sector causes when sectoral weights and labour shares covary, as I will show in the next section using the industry-level data.

% Empirically, the cross term is large and negative, as I show in the next section. Using the shift-share decomposition measures a zero reallocation effect, but the Haltiwanger decomposition recovers a positive effect. Moreover, the shift-share decomposition overcounts the negative within-sector contribution. 

Furthermore, the Haltiwanger method is preferable to the shift-share method when there are trends in either $\lambda_{it}$ or $\omega_{it}$. For instance, a downward trend in $\lambda_{it}$ pushes the arithmetic mean, $\tilde{\lambda}_{it}$, down each period, meaning the shift-share between term captures reallocation and sector-level trends in labour shares (a within-sector phenomenon). \citet{diez-catalanLaborShareService2018, elsbyDeclineLaborShare2013a} and \citet{gutierrezInvestigatingGlobalLabor2017} show the labour share is trending down in virtually every sector and rising in a few. The presence of sectoral trends does not pose an issue when using the Haltiwanger decomposition because the bases are fixed at the previous period values. 

By examining only the shift-share and Haltiwanger decompositions, I omit several other commonly used decomposition techniques. For example, changes in the aggregate labour share can also be decomposed as in  \citet{olleyDynamicsProductivityTelecommunications1996}. I use the shift-share method since it is typically used in studies that decompose the declining labour share. Moreover, I use the Haltiwanger method because, as Theorem \ref{theorem_1} shows, it is closely related to the shift-share method, but additionally accounts for co-movements between sectors' weights and labour shares.  




